% Copyright 2023 The terCAD team. All rights reserved.
% Use of this content is governed by a CC BY-NC-ND 4.0 license that can be found in the LICENSE file.

\subsection{Creating Skeleton}
\markboth{Prototyping}{Creating Skeleton}

In case of uncertainties, the problem resolution is started from a prototype creation with "no architecture" approach 
and sometimes even with "no code" solution. Those first steps are mostly done by Designers and Researches (Business 
Analysts) that collaborate together to form initial conditions by interpreting users' pain points into features.

As an example, Dropbox founder Drew Houston explained that a demo video was released during Dropbox's early days
and has helped to form a partnership with Arash Ferdowsi to proceed with its farther creation, "it was easy to explain 
the idea, but really hard to do it" \cite{Shon13}.

Without the luxury of a dedicated team and abundant resources, similar to the indie game development model, we might 
shift our focus toward becoming both consumers and contributors. Creating a product and assuming that everyone grasps 
its functionality is a misconception disconnected from reality. What seems straightforward to us might pose a challenge 
for someone else, and that's perfectly normal. Each individual has their unique perspective, and what appears effortless 
to them can be perceived as intricate by others. That highlights the fact that the application has to be configurable 
and adaptive.

Let's begin by identifying preferences and priorities from our fillings, focusing on what holds the utmost significance 
for us on the initial screen, and a secondary information within the "one-click" vicinity. We should avoid being overly 
influenced by existing applications, as doing so could lead us down the path of mimicry (term, "create own black-jack"), 
or potentially stifle our initial impulse and creative drive (term, "re-implement the wheel"). This approach aligns 
with the Emotional Cycle of Change \cite{Pfei79}, transitioning from "Uninformed optimism" as we start, to 
"Informed pessimism" after achieving our initial working solution or minimum viable product (MVP) by stepping into 
competitors analysis.

A \textbf{Financial Accounting Application} should display our current balance (particularly essential in cases of 
multi-banking), budget limits for various expense categories, a history of expenses (especially for a shared mode of 
the application usage, as between partners), and visualize our financial goals, which serves as the core motivation for 
using the application. Secondly, it's important to view and analyze income and expense metrics, that are giving insights 
into our financial patterns. 

Once we've formulated an idea of \q{what} to showcase but find ourselves challenged in understanding \q{how} to 
represent it, let's take a peek at any of existing apps, as "Rally" in Flutter Gallery 
(\href{https://gallery.flutter.dev/\#/rally}{https://gallery.flutter.dev/\#/rally}, \cref{img:proto-rally}).
The insightful overview revealed several key design considerations. It highlighted the effectiveness of using a 
vertical bar to visually represent the remaining budget within a category. Furthermore, it showcased the versatility of 
headers, not only for grouping content but also for summarizing information. Additionally, the strategic use of colors 
and fonts facilitated the connection of data points, such as linking bills with their corresponding budget categories. 

That has helped us to formalize that the core interface can be effectively constructed using only two essential 
components: "Title" and "Item". In our context, the \q{BaseWidget} is designed to combine a header with a list of items, 
while the \q{BaseLineWidget} serves to represent individual items as lines \issue{2}{}. Fortunately, Flutter's extensive 
library of components unlocks a prototyping process from the code.

\img{prototyping/rally-app}{Rally application from Flutter Gallery}{img:proto-rally}


\subsubsection{Assembling Blocks}

First and foremost, it's crucial to comprehend how to make the most efficient use of the available screen real estate.

To maximize the utilization of the available screen space in terms of both height and width, it is vital to make use of 
the \q{LayoutBuilder}-widget. This widget provides access to the constraints of its parent widget, enabling us to 
determine the maximum available height for its child widgets:

\begin{lstlisting}
LayoutBuilder(
  builder: (BuildContext context, BoxConstraints constraints) {
    final maxHeight = constraints.maxHeight;
    return Container(
      color: Colors.blue,
      height: maxHeight,
      child: Column(
        children: [
          Container(color: Colors.red, height: maxHeight * 0.5),
          Container(color: Colors.green, height: maxHeight * 0.3),
          Container(color: Colors.yellow, height: maxHeight * 0.2),
        ],
      ),
    );
  },
);
\end{lstlisting}

\noindent In the example above, we wrap the nested \q{Container}-widgets with a \q{LayoutBuilder}-widget. Inside the 
\q{builder} function, we retrieve the \q{maxHeight}-property from the \q{constraints} provided by the \q{LayoutBuilder}. 
That helps us to dynamically calculate and assign the maximum height to the nested \q{Container}-widgets based on the 
available space within their parent widget. And... that leads to our first error: 

\begin{lstlisting}[language=terminal]
The following assertion was thrown during performLayout():
  BoxConstraints forces an infinite width.

  RenderFlex children have non-zero flex but incoming width 
  constraints are unbounded. When a row is in a parent that does 
  not provide a finite width constraint, for example if it is in 
  a horizontal scrollable, it will try to shrink-wrap its children 
  along the horizontal axis. Setting a flex on a child (e.g. using 
  Expanded) indicates that the child is to expand to fill the 
  remaining space in the horizontal direction.
\end{lstlisting}

\noindent To address the error, we might either use a combination of \q{Expanded} and \q{FractionallySizedBox} widgets,
or bound the \q{LayoutBuilder} by \q{SizedBox} or \q{Container} with defined height, or by using \q{Expanded} as a root 
element.

\begin{lstlisting}
LayoutBuilder(
  builder: (BuildContext context, BoxConstraints constraints) {
    final maxHeight = constraints.maxHeight;
    return Container(
      color: Colors.blue,
      child: Column(
        children: [
          Expanded(
            child: FractionallySizedBox(
              alignment: Alignment.topCenter,
              heightFactor: 0.5,
              child: Container(color: Colors.red),
            ),
          ),
          Expanded(
            child: FractionallySizedBox(
              alignment: Alignment.topCenter,
              heightFactor: 0.3,
              child: Container(
                color: Colors.green,
              ),
// ... other content
\end{lstlisting}

\noindent We may create a vertical indentation between elements by using either the \q{Padding}-widget or an "empty" 
\q{SizedBox} (without a "child" property):

\begin{lstlisting}
SizedBox(height: 16.0), // Vertical spacing of 16.0 DIP
\end{lstlisting}

\noindent To take the maximum available width for a widget we would use \q{Expanded}-widget (or, it can be 
\q{Flexible}-widget) that allows nested widgets to expand and take up all the remaining space within a row or column, 
also, it makes them scrollable in case of exceeding the limitation. Another approach -- by an infinitive width 
declaration (\emph{additionally, \q{OverflowBox}-widget allows to control how the child widget behaves when it 
exceeds the available space}):

\begin{lstlisting}
OverflowBox(
  maxWidth: double.infinity, // To take the maximum allowed width
  child: ContentWidget(),
);
Container(
  width: double.infinity, // To make full-sized button
  child: FloatingActionButton(
    onPressed: () {}, // Handle button press
    child: Icon(Icons.add),
  ),
);
\end{lstlisting}

\noindent To create a vertical progress bar, we'll proceed with the usage of a \q{LinearProgressIndicator}-widget and 
rotate it vertically (-90$^{\circ}$, counter-clockwise) by using the a \q{Transform}-widget:

\begin{lstlisting}
// ./lib/routes/home/baseListWidget.dart
class BaseLineWidget extends StatelessWidget {
  /* ... properties declaration */
  const BaseLineWidget({super.key, required this.title, /*...*/ });
  @override
  Widget build(BuildContext context) {
    return Column( // To group content with a divider at the end
      crossAxisAlignment: CrossAxisAlignment.start,
      children: [
        Row( // Spread apart Description and Amount
          mainAxisAlignment: MainAxisAlignment.spaceBetween,
          children: [
            Row( // Left Side Section (Info)
              children: [
                BarVerticalSingle(
                  value: progress, 
                  height: 24, 
                  color: color,
                ),
                Column(
                  crossAxisAlignment: CrossAxisAlignment.start,
                  children: [
                    // The main information
                    Text(title, style: TextStyle(fontSize: 16, 
                        fontWeight: FontWeight.bold)),
                    // Additional info
                    Text(description,
                      style: TextStyle(fontSize: 12),
                    ),
                  ],
                ),
              ]),
            Padding( // Right-Side Block (Amount)
              padding: const EdgeInsets.all(8.0),
              child: Text(details),
            ),
          ]),
        Divider(), // To plot a horizontal line
      ]);
  }(*@ \stopnumber @*)
}

// ./lib/charts/barVerticalSingle.dart
class BarVerticalSingle extends StatelessWidget {
  /* ... properties declaration */
  const BarVerticalSingle({super.key, /* ... */ });
  @override
  Widget build(BuildContext context) {
    return Container(
      width: height,
      height: width,
      child: Transform.rotate(// Rotate vertically
        angle: -90 * pi / 180,
        child: LinearProgressIndicator(
          value: value,
          backgroundColor: Colors.grey, // Background color
          valueColor: AlwaysStoppedAnimation<Color>(color),
// .. closing brackets
\end{lstlisting}

\noindent As a note, the rotation can be improved by the \q{RotatedBox}-widget usage.

\img{prototyping/prototype-v1}{The initial result stemming from our implementation}{img:prototype}

\begin{lstlisting}
// ./lib/routes/home/baseWidget.dart
class BaseWidget extends StatelessWidget {
  final String title;
  const BaseWidget({super.key, required this.title});
  @override
  Widget build(context) {
    return Expanded(
      child: Container(
        margin: EdgeInsets.all(8.0),
        child: Column(
          crossAxisAlignment: CrossAxisAlignment.start,
          children: [
            Container(
              color: Theme.of(context).colorScheme.inverseSurface,
              child: Column(
                crossAxisAlignment: CrossAxisAlignment.start,
                children: [
                  Text(title, style: textTheme.labelSmall),
                  Text('123,456.789', // hardcoded number
                    style: textTheme.displaySmall,
                  ),
                ],
              ),
            ),
            // To accumulate the remaining height
            Expanded( 
              child: ListView(
                children: [
                  BaseLineWidget(
                    title: 'Description $title 1',
                    description: '2023-06-17',
                    details: formatter.format(12345.789),
                    progress: 0.8,
                  ),
                  // ... other blocks
// ... closing brackets
\end{lstlisting}

\noindent As it might be noticed, we've injected our data into components. This helps to concentrate on design 
aspects (\cref{img:prototype}) rather than getting bogged down (by formulating) in an unfamiliar data structure. 
However, preserving the data injection per each component could also be counterproductive, as it would necessitate 
extra efforts to cleanup the components later. That's why we should separate representation and data layers on the 
early stages independently of understanding the context of a data structure \issue{10}{}:

\begin{lstlisting}[caption=Mock Structure of a Financial Data, label=code:app-data]
// ./lib/data.dart
class AppData {
  final _data = {
    'goals': [(
      title: 'Implement new functionality to reach the goal',
      startDate: '2022-01-01 00:00',
      endDate: '2024-09-01 00:00',
    )],
    'bills': {
      'total': 123456.789,
      'list': [(
          details: 12345789.098,
          // Approach to split String into several chunks
          title: 'Description with a long explanation'
              '... long explanation and details',
          description: '2023-06-17',
          progress: 0.5,
          color: Colors.red,
        ), // .. other sections and data
  };
  Map<String, dynamic> get state => _data;
  void set(String property, dynamic value) {
    if (!_data.containsKey(property)) throw ArgumentError('...');
    _data[property] = value;
  }
}
\end{lstlisting}

\noindent By separating information from its representation, we need an option to iteratively generate widgets:

\begin{lstlisting}
ListView(
  children: myList.map((item) {
    return Padding( // indentation widget
      padding: EdgeInsets.symmetric(horizontal: 16.0),
      child: ListTile(title: Text(item)),
    );
  }).toList(),
);
\end{lstlisting}

\noindent In the example above, we have a \q{ListView} with a list of items stored in the \q{myList}-variable. We use 
the \q{map}-method to iterate over each item in the list and return a \q{Padding}-widget wrapped around a 
\q{ListTile} (a single fixed-height row with a capability of adding leading or/and trailing icons). Finally, we convert 
the resulting output \q{Iterative<Widget>} to a list of children (\q{List<Widget>}) using the \q{toList}-method. 

\begin{lstlisting}
Expanded(
  child: ListView.builder(
    itemCount: state['list'].length + 2, //+ leading and trailing
    // To inject widget in between with 'ListView.separated' usage
    /// separatorBuilder: (context, index) => SizedBox(height: 4.0),
    // To improve the rendering performance from a template usage
    /// prototypeItem: ListTile(title: Text(state['list'].first)),
    itemBuilder: (context, index) {
      // Leading indentation
      if (index == 0) { 
        return SizedBox(height: 8.0);
      } else if (index <= state['list'].length) {
        final item = state['list'][index - 1];
        return BaseLineWidget(
          title: item.title,
          description: item.description,
          details: item.details,
          progress: item.progress,
          color: item.color,
        );
      // Trailing "button" at the end
      } else { 
        return TextButton(
          onPressed: () {}, // Handle "More" button pressed
          child: Text(AppLocalizations.of(context)!.btnMore),
        );
      }
    }),
);
\end{lstlisting}

\noindent It's the first time when we've used a localization (line 25, autogenerated \q{AppLocalizations}-class) but the 
discussion will be postponed till \cref{locale}. Its usage matches with our goal of separating the presentation and data 
layers (\emph{as a tip, they should not be transpiled into the folders structure, organize files around the targets}):

\hspace{0.5cm}
\dirtree{%
  .1 Application.
  .2 Business Logic Layer.
  .3 Functional Rules.
  .3 Computations.
  .3 Data Structures.
  .2 Service Layer \emph{(intermediate layer)}.
  .3 Security Layer (Controllers).
  .3 Extensions \emph{(extend, unify or wrap basics)}.
  .2 Data Access Layer.
  .3 Representation Layer \emph{(a visual state of primitives)}.
  .3 Connectors \emph{(connection to different sources)}.
  .3 Subscribers \emph{(to notify about the data change)}.
  .2 Presentation Layer.
  .3 Communication Layer \emph{(exchange data between visual elements)}.
  .3 User Experience Layer \emph{(reflect to the context of usage)}.
  .3 Interface Layer \emph{(basic building blocks, pure UI modules)}.
  .3 Components \emph{(inject data into their respective interfaces)}.
  .3 Routing \emph{(respond to user interactions)}.
}
\hspace{1cm}

\noindent Described iterative builder can be additionally improved by the usage of \q{Stream}s (instead of \q{List}s, 
\issue{305}{}) and \q{KeepAliveWrapper}-widget (\emph{with a \q{AutomaticKeepAliveClientMixin}-mixin that retains the 
state of widgets that are either expensive to rebuild, or come in and out of the view, -- \q{ListView} and 
\q{GridView}}, \issue{86}{}) in the context of an optimal memory allocation and utilization:

\begin{lstlisting}
class KeepAliveWrapper extends StatefulWidget {
  final Widget child;
  const KeepAliveWrapper({super.key, required this.child});
  @override
  KeepAliveWrapperState createState() => KeepAliveWrapperState();
}

class KeepAliveWrapperState extends State<KeepAliveWrapper> 
    with AutomaticKeepAliveClientMixin {
  @override
  bool get wantKeepAlive => true;
  @override
  Widget build(BuildContext context) {
    super.build(context); // required by mixin
    return widget.child;
  }
}
\end{lstlisting}


\subsubsection{Separating Context}

\noindent While the main page is ready, let's discuss routing in Flutter \issue{14}{}:

\begin{lstlisting}
// ./lib/routes.dart
const String homeRoute = '/app/finance';
const String accountRoute = '/app/finance/account';
const String accountAddRoute = '/app/finance/account/add';
const String accountViewRoute = '/app/finance/account/uuid:';(*@ \stopnumber @*)
const String accountEditRoute = '/app/finance/account/uuid:/edit';

// ./lib/main.dart
void main() => runApp(const MyApp());

class MyApp extends StatefulWidget {
  const MyApp({super.key});

  @override
  _MyAppState createState() => _MyAppState();
}

class _MyAppState extends State<MyApp> {
  final AppData state = AppData();

  MaterialPageRoute? getDynamicRouter(settings) {
    final String route = settings.name!;
    final regex = RegExp(r'\/uuid:([\w-]+)');
    final match = regex.firstMatch(route);
    if (match != null) {
      final String uuid = match.group(1) ?? '';
      switch (route.replaceAll(uuid, '')) {
        case routes.accountViewRoute:
          return MaterialPageRoute(builder: (context) => 
                AccountViewPage(state: state, uuid: uuid));
        case routes.accountEditRoute:
          return MaterialPageRoute(builder: (context) => 
                AccountEditPage(state: state, uuid: uuid));
      }
    }
    return null;
  }

  @override
  Widget build(BuildContext context) {
    return MaterialApp(
      initialRoute: routes.homeRoute,
      onGenerateRoute: getDynamicRouter,
      routes: <String, WidgetBuilder>{
        routes.homeRoute: (context) => HomePage(state: state),
        routes.accountRoute: (context) => AccountPage(state: state),
// ... other content
\end{lstlisting}

\noindent In the current scenario, we utilize the \q{onGenerateRoute}-method to manage routes with a dynamic structure. 
This involves extracting the \q{uuid}-parameter from the route path and return the appropriate widget with propagated 
parameters. For more "static" content the \q{routes}-list is used. It's essential to note that while this approach 
aligns with URLs construction patterns, Flutter doesn't treat routes as unique URLs in the browsing history. Therefore, 
it's more practical to work with additional parameters instead of \q{RegExp} transformations \issue{249}{}:

\begin{lstlisting}
MaterialPageRoute? getDynamicRouter(RouteSettings settings) {
  final args = settings.arguments as Map<String, String>?;
  final uuid = args?['uuid'] ?? '';
  final route = settings.name ?? '';(*@ \stopnumber @*)
  // ...
// To call the route with parameters
Navigator.pushNamed(context, routeName, arguments: {'uuid': '1'});
\end{lstlisting}

\noindent Additionally, it's worth considering removing the \q{routes}-attribute from the \q{MaterialApp}-widget. This 
enables the usage of dynamic construction for all routes, unifying the control over their management.

\begin{lstlisting}
final routes = <String, WidgetBuilder>{
  AppRoute.accountViewRoute: (context)=> AccountViewPage(uuid: key),
  // ... other pages
  AppRoute.subscriptionRoute: (context) => const SubscriptionPage(),
};
return routes[route] ?? (context) => const HomePage();
\end{lstlisting}

\noindent With the routing declaration now finalized, it's time to proceed with the creation of a navigation menu. To 
add buttons to the \q{AppBar}-widget, we may use the \q{actions}-property, which allows us to supply a list of 
widgets representing the buttons that we'd like to place on the right side, and the \q{leading}-property, to utilize 
the left corner space:

\begin{lstlisting}
Scaffold(
  appBar: AppBar(
    title: Text('AppBar with Button'),
    actions: [
      IconButton(icon: Icon(Icons.settings), onPressed: /*...*/), 
\end{lstlisting}

\noindent We may additionally customize the appearance and behavior of buttons by using their different types, as 
\q{IconButton}, \q{FlatButton}, \q{RaisedButton}, \q{TextButton}, or \q{PopupMenuButton}:

\begin{lstlisting}
AppBar(
  title: Text('My App'),
  actions: [
    PopupMenuButton(
      icon: Icon(Icons.more_vert),
      onSelected: (value) {/* ... */},
      itemBuilder: (BuildContext context) {
        return [
          PopupMenuItem(
            child: Text('Option 1'),
            value: 'option1',
          ),
          // ... other options
// ... closing brackets
\end{lstlisting}

\noindent Each menu item has a child widget, which in this case is a \q{Text}-widget, and a value associated with it. 
When a menu item is selected, the \q{onSelected}-callback is triggered with the selected value.

There are multiple options available for creating the navigation. As \q{Drawer}-widget offers a panel that slides in 
from either the left or right side of the screen to present navigation options or settings:

\begin{lstlisting}
class MyApp extends StatelessWidget {
  void _navigateToPage(context, String routeName) {
    // Trick to close the drawer
    Navigator.pop(context); 
    // Navigate to the specified route
    Navigator.pushNamed(context, routeName); 
  }
  @override
  Widget build(BuildContext context) {
    return MaterialApp(
      home: Scaffold(
        appBar: AppBar(
          title: Text('Navigation Menu Example')),
        drawer: Drawer(
          child: ListView(
            padding: EdgeInsets.zero,
            children: [
              DrawerHeader(
                decoration: BoxDecoration(color: Colors.blue),
                child: Text(
                  'Navigation Menu',
                  style: TextStyle(fontSize:24, color:Colors.white),
                )),
              ListTile(
                leading: Icon(Icons.home),
                title: Text('Home'),
                onTap: () => _navigateToPage(context, '/home'),
              ),
// ... other options
\end{lstlisting}

\noindent Each \q{ListTile} serves as a navigation item and is attributed an icon, title, and an \q{onTap}-callback. 
For customizing the appearance of the \q{Drawer}, we may eliminate the rounded corners and give it a more squared-off 
appearance:

\begin{lstlisting}
Drawer(
  // Remove the elevation (shadow) of the drawer
  elevation: 0,
  // Disable rounded corners
  shape: Border.all(width: 0), 
  child: Container(
    // Set the background color
    color: Colors.grey[900], 
    child: /* ... */
  ),
);
\end{lstlisting}

\noindent For a desktop application it's important to reflect a mouse hover (covered by a \q{MouseRegion}-widget): 

\begin{lstlisting}
class MyDrawer extends StatelessWidget {
  final List<String> drawerItems = [/* ... */];

  @override
  Widget build(BuildContext context) {
    return Drawer(
      child: ListView.builder(
        itemCount: drawerItems.length,
        itemBuilder: (context, index) {
          return MouseRegion(
            cursor: SystemMouseCursors.click,
            onEnter: (event) { /* ... */ },
            onExit: (event) { /* ... */ },
            child: ListTile(
              title: Text(drawerItems[index]),
              onTap: () { /* ... */ },
// ... other options
\end{lstlisting}

\noindent \q{InkWell}-widget might be an alternative to the \q{MouseRegion}-widget usage.


\newpage
\subsubsection{Configuring Inputs}

Going further, the defined setter in our fake data storage (\cref{code:app-data}) would help to implement 
\q{Add} and \q{Edit} forms \issue{16}{bed81f5} (\cref{img:proto-form}). Flutter provides a set of widgets to display all 
basic inputs as \q{TextFormField} (for input), \q{DropdownButton} / \q{DropdownMenuItem} and \q{SearchAnchor} (for 
selector), \q{Switch} (toggler), and others. Additionally, Package Manager (\q{pub}) can be used to apply additional
components, as \q{Icon}-picker (\q{flutter\_iconpicker}-package), \q{Color}-selector (\q{flutter\_colorpicker}-package),
and others. However, after combining dozens of different widgets from various libraries, we might encounter a stylistic 
inconsistency between these widgets (term, "UX Hell") and end up having to make compromises or even replace some of them
by the original components afterwards (\issue{149}{}, \issue{200}{}, \issue{204}{}, \issue{207}{}, \issue{329}{}).

That's why it's important to highlight that all the inputs (we've used on the form) have been encapsulated within 
their own widget (\q{SimpleInput}, \q{MonthYearInput}, \q{IconSelector}, etc.). This approach facilitates a seamless 
transition from one presentation layer to another or from one external library to another, all without requiring any 
changes beyond the component itself \footnote{This technique finds application in the majority of Object-Relational 
Mapping (ORM) libraries, where its primary purpose is to standardize the interface for connecting to and interacting 
with various databases. By abstracting the underlying complexities of different database systems, ORM libraries 
streamline the process of database access and data manipulation, making it easier for developers to work with diverse 
database platforms in a generalized manner.}.

By starting with a basic input field, it's essential to ensure that the input matches our expectations, such as 
expecting numbers for numeric fields. To prevent users from typing anything outside the defined \q{TextInputType}, 
we should create a custom \q{TextInputFormatter} and attach it to the \q{inputFormatters}-property:

\begin{lstlisting}
class LimitedTextInputFormatter extends TextInputFormatter {
  final TextInputType textInputType;
  const LimitedTextInputFormatter(this.textInputType);

  @override
  TextEditingValue formatEditUpdate(
      TextEditingValue oldValue, TextEditingValue newValue) {
    // Return the old value if the new value is not valid
    return (newValue.text.isNotEmpty &&
        !_isValidInput(newValue.text, textInputType)) ?
      oldValue : newValue;
  }
  bool _isValidInput(String text, TextInputType textInputType) {
    return switch (textInputType) {
      TextInputType.number => double.tryParse(text) != null;
      TextInputType.phone => 
          text.contains(RegExp(r'^[0-9+\-()\s]+$'));
      // ... more cases for other types
\end{lstlisting}

\noindent An example of the formatter usage:

\begin{lstlisting}
TextFormField(
  inputFormatters: [
    LimitedTextInputFormatter(textInputType),
\end{lstlisting}

\noindent Alternatively, the allowance can be declared via a \q{RegExp}-expression:

\begin{lstlisting}
TextFormField(
  // Force the keyboard to show a decimal point button
  keyboardType: TextInputType.numberWithOptions(decimal: true),
  inputFormatters: [ // Allow numbers with at max two decimal places
    FilteringTextInputFormatter.allow(RegExp(r'^(\d+)?\.?\d{0,2}')),
  ],
  // ... other options
);
\end{lstlisting}

\noindent And, for the case of forbidding everything, we may use the boolean \q{readOnly}-property. 

It's important to mention that the changes in the field's value are managed and monitored through \q{TextEditingController}. 
That component allows bidirectional communication with text input widgets. It does not only enable retrieving the 
current value of a text field but also provides a capability to programmatically change the content (whether a real-time 
validation, or respond to user inputs):

\begin{lstlisting}
class CurrencySelector extends StatelessWidget {
  final _controller = TextEditingController();
  final Function callback;
  final Currency? value;

  CurrencySelector({
    super.key,
    this.value,
    required this.callback,
  }) {
    _controller.text = value.symbol;
  }

  @override
  Widget build(BuildContext context) {
    final color = Theme.of(context).colorScheme.primary;
    return TextFormField(
      key: ValueKey(value),
      controller: _controller,
      decoration: InputDecoration(
        filled: true,
        border: InputBorder.none,
        fillColor: color.withValues(alpha: 0.1),
        suffixIcon: GestureDetector(
          onTap: () => showCurrencyPicker(
              context: context,
              showFlag: true,
              showCurrencyName: true,
              showCurrencyCode: true,
              onSelect: (Currency currency) => callback(currency),
            ),
          child: const Icon(Icons.arrow_drop_down),
// ... closing brackets
\end{lstlisting}

\noindent In this code, we set the key property of the \q{TextFormField} to \q{ValueKey(value)} to ensure that when the 
state changes and the \q{CurrencySelector} is rebuilt, the \q{TextFormField} will receive a new key based on the updated 
value, -- force the \q{TextFormField} to re-render.

For the \q{Color}-picker might be added a tiny improvement as a color preset (randomly taken via 
\q{Color.getRandomMaterialColor()}-call):

\begin{lstlisting}
// ./lib/_ext/color_ext.dart
extension ColorExt on Color {
  MaterialColor get toMaterialColor {
    final Map<int, Color> colorMap = {};
    for (int i = 50; i <= 900; i += 100) {
      colorMap[i] = Color.fromRGBO(red, green, blue, i / 1000);
    }
    return MaterialColor(value, colorMap);
  }
  static Color getRandom() {
    List<Color> colors = Colors.primaries;
    Random random = Random();
    return colors[random.nextInt(colors.length)];
  }
  static MaterialColor getRandomMaterialColor() {
    return getRandom().toMaterialColor;
  }
}
\end{lstlisting}

\noindent The \q{getRandom}-function selects a random color from the \q{Colors.primaries}-list using the 
\q{Random}-class; the \q{MaterialColor}-constructor requires shades of the color, that's why we've created a \q{Map} 
with shade values ranging from 50 to 900.

An icon on the opposite side of the text "Balance Date Update" (\cref{img:proto-form}), mostly known as an informational 
anchor, should show a tooltip by a long-press. To add such a behavior to the icon, it should be wrapped by a \q{Tooltip} 
or \q{Semantics}-widget:

\begin{lstlisting}
Tooltip(message: 'This is a tooltip', child: /* ... */);
\end{lstlisting}

\img{prototyping/prototype-form}{Form to create new Account}{img:proto-form}

\noindent Also, our form may exceed the available vertical space, and in that case we should use a
\q{SingleChildScrollView}-widget (or, \q{CustomScrollView}) to make the content scrollable. On the other hand, 
it might be needed to hide elements based on a screen resolution. That can be achieved by using a \q{MediaQuery}-widget 
(or, constraints sizes) to access the screen dimensions and conditionally render the content based on a specific 
threshold.

\begin{lstlisting}
final minHeightThreshold = 640;
final minWidthThreshold = 480;

LayoutBuilder(
  builder: (BuildContext context, BoxConstraints constraints) {
    double screenHeight = MediaQuery.of(context).size.height;
    /// double maxWidth = constraints.maxWidth;
    /// double maxHeight = constraints.maxHeight;
    if (screenHeight < minHeightThreshold) {
      return Container(); // Mimic a 'Visibility'-widget
    } else {
      return ListView(
        children: <Widget>[/* ... */],
      );
    }
  },
);
\end{lstlisting}

\noindent This approach allows us to conditionally render the widget based on the screen height, providing a responsive
and adaptable user interface (UI). Furthermore, \q{MediaQuery} helps to identify not only sizes of the screen, but
an orientation of it:

\begin{lstlisting}
MediaQuery.of(context).orientation == Orientation.landscape
\end{lstlisting}

\noindent As for the text, it can be truncated with an ellipsis (...) if it overflows the defined container 
constraints:

\begin{lstlisting}
Container(
  constraints: BoxConstraints(
    maxWidth: 200,
    minWidth: 50,
  ),
  /// width: 200,
  child: Text(
    'This is a long text that may overflow the container',
    maxLines: 1,
    overflow: TextOverflow.ellipsis,
  ),
)      
\end{lstlisting}
